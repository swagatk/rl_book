\chapter{Actor-Critic Models}
\index{Actor-Critic}
\gls{ac} models are a class of reinforcement learning (RL) algorithms that combine the strengths of
both policy-based (actor) and value-based (critic) methods. This hybrid approach aims to address the
limitations of using each method individually. It has two components, namely, actor and critic. The
actor is responsible for selecting actions based on the current state. It learns a policy function,
often parameterized by $\theta$, which maps states to actions or a probability distribution over
actions. The actor's goal is to improve its policy to maximize the expected cumulative reward. The
critic's role is to evaluate the actions taken by the actor. It learns a value function, typically
parameterized by $\phi$, which estimates the expected future rewards for a given state (state-value
function, $V(s)$) or state-action pair (action-value function, $Q(s,a)$). The critic provides
  feedback to the actor on how "good" its actions were.

Actor-critic models are motivated by the desire to combine the advantages of policy-based and
value-based methods while mitigating their individual limitations. Policy-based methods (e.g.
REINFORCE) can handle continuous action spaces, learn stochastic policies (which is useful for
exploration and in non-deterministic environments), and can directly optimize the policy. However,
they suffers from high variance in gradient estimates because they rely on full episode returns.
This high variance can lead to unstable training and slow convergence. On the other hand,
value-based methods (e.g. Q-learning, SARSA) generally have low variance due to bootstrapping (using
estimated values to update other estimated values). They are often more sample-efficient. However,
these methods can only be applied to problems with discrete action spaces and typically learn
deterministic policies. Actor-critic methods aim to gain the best of both worlds: the ability to
handle continuous action spaces and learn stochastic policies (from the actor) combined with the
reduced variance and improved stability from bootstrapping with a learned value function (from the
critic).

The use of actor-critic model in implementing algorithms such DDPG and PPO have already been
discussed in the previous chapter. In this chapter, we will discuss the common architectures in more
details. 

\section{Naive Actor-Critic Model}\label{sec:naive_ac}
\index{Actor-Critic!Naive}
It is a simple version of actor-critic methods where the critic estimates the value function
$V(s;w)$, and the actor updates the policy using the critic’s feedback, typically via policy
gradients. The actor aims to maximize the expected cumulative reward, given by:
\begin{equation}
	J(\theta) = \mathbb{E}_{\pi_\theta} [G_0] 
	\label{eq:nac_cost}
\end{equation}
where $G_0$ is the total discounted return from the initial state with:
\begin{equation}
	G_t = \left[\sum_{k=0}^{\infty}\gamma^kR_{t+k+1}\right]
	\label{eq:nac_disc_return}
\end{equation}
Here, $\gamma \in[0, 1]$ is the discount factor, and $R_t$ is the reward received at time t. The actor updates its policy parameters $\theta$ by performing gradient ascent on this objective function. The Policy Gradient Theorem provides the fundamental basis for this update. A practical form of the gradient of $J(\theta)$ used for updating policy parameters is given by:
\begin{equation}
	\nabla_\theta J(\theta) = \mathbb{E}_{\pi_\theta}[\nabla_\theta\log\pi_\theta(a|s)Q^\pi(s,a)]
	\label{eq:nac_pg_update}
\end{equation}


\subsection{Temporal Difference (TD) Error}
In the Naive Actor-Critic algorithm, we approximate $Q^\pi(s,a)$ with the critic's estimate
$Q_w(s,a)$ (or $V_w(s)$) along with the immediate reward. The critic learns its value function by
minimizing the difference between its current estimate and a more "bootstrapped" estimate of the
value. This difference is known as the Temporal Difference (TD) error. For a state-value function
$V_w(s)$, the one-step TD error is:

\begin{equation}
	\delta_t = R_{t+1} +\gamma V_w(S_{t+1}) - V_w(S_t)
	\label{eq:nac_td_error}
\end{equation}
where $S_t$ is the state at time $t$, $A_t$ is the action taken, and $R_{t+1}$ is the reward
received after taking action $A_t$ and transitioning to $S_{t+1}$.

If the critic estimates an action-value function $Q_w(s,a)$, the TD error (similar to SARSA) would be: 
\begin{equation}
	\delta_t = R_{t+1} + \gamma Q_w(S_{t+1}, A_{t+1}) - Q_w(S_t,A_t)
	\label{eq:nac_td_error_q}
\end{equation}
The TD error serves as the signal for both the critic and the actor. 

\subsection{Advantage Function} \label{sec:nac_adv}
To improve the stability and performance of policy gradient methods, the concept of an advantage
function is often used. The advantage function $A^\pi(s,a)$ measures how much better an action $a$
is compared to the average value of the state s under policy $\pi$:

\begin{equation}
	A^\pi(s,a) = Q^\pi(s,a) - V^\pi(s)
	\label{eq:nac_adv}
\end{equation}
When using the advantage function, the policy gradient update becomes:
\begin{equation}
	\nabla_\theta J(\theta) = \mathbb{E}_{\pi_\theta}[\nabla_\theta\log\pi_\theta(a|s)A^\pi(s,a)]
	\label{eq:nac_pg_update_adv}
\end{equation}
In the Naive Actor-Critic, the TD error itself can be seen as an estimate of the advantage function.
This is because $R_{t+1}+\gamma V_w(S_{t+1})$ can be viewed as a noisy estimate of $Q_w(S_t, A_t)$
for the taken action, and $V_w(S_t)$ is the baseline. So, $\delta_t \approx Q_w(S_t,A_t) -
V_w(S_t)$. The pseudocode for naive actor-critic algorithm is provided in Table \ref{alg:nac}. 

\begin{algorithm}[htbp]
	\caption{Actor-Critic Algorithm}
	\label{alg:nac}
	\begin{algorithmic}[1]
		\State input: Policy parameters $\theta$, value function parameters $w$. Learning rates $\alpha_\theta$, $\alpha_w$, discount factor $\gamma$.
		\For {each episode}
		\State Initialize $s_t$
		\While{episode not done}
		\State Sample action $a_t \sim \pi(a|s;\theta)$
		\State Take action $a_t$, obtain reward $r_t$, transition to new state $s_{t+1}$
		\State Compute TD error: $\delta_t = R_{t+1} +\gamma V(S_{t+1};w) - V(s_t;w)$
		\State Update Critic: $w \leftarrow w + \alpha_w \delta_t \nabla_w V(s_t;w)$ 	\Comment{Critic Update}
		\State Update Actor: $\theta \leftarrow \theta + \alpha_\theta \delta_t \nabla_\theta \log \pi(a|s;\theta)$ \Comment{Actor Update}
		\State $s_t \leftarrow s_{t+1}$
		\EndWhile
		\EndFor
	\end{algorithmic}
\end{algorithm}

\subsection{Advantages of Actor-Critic Algorithm}
Actor critic methods can provide the following advantages in general:
\begin{itemize}
\item Reduced Variance: Compared to pure policy gradient methods (like REINFORCE), the critic's
  value estimate acts as a learned baseline, significantly reducing the variance of the policy
  gradient estimate. 
\item Continuous Action Spaces: Actor-critic methods can easily handle continuous action spaces by
  having the actor output parameters of a continuous probability distribution.

\item On-policy Learning: The critic learns about the policy currently being followed by the actor,
  providing more relevant feedback.
\end{itemize}

\subsection{Limitations of Naive Actor-Critic Algorithm}
\begin{itemize}
   \item Bias from Critic Approximation: If the critic's value estimate is inaccurate, it can
     introduce bias into the policy gradient, leading to suboptimal policies.  
   \item Convergence Issues: The interaction between the actor and critic can be unstable, sometimes
     leading to convergence problems.  
   \item Sample Inefficiency: While better than pure policy gradient, it
     can still be sample-inefficient compared to off-policy methods.  
 \end{itemize} 

 The ``naive'' actor-critic algorithm generally uses a $Q(s,a)$ or $V(s)$ estimate directly for
 actor update as given below:
 \begin{equation}
   \theta \leftarrow \theta + \alpha_\theta \nabla_\theta \log \pi(a|s;\theta) V(s_t;w)   
   \label{eq:nac_actor_update}
 \end{equation}
This, however, can be quite unstable and may exhibit high variance. Therefore, TD error is used for
updating actor parameters to reduce the variance. The resulting actor update equation is given by
\begin{equation}
   \theta \leftarrow \theta + \alpha_\theta \nabla_\theta \log \pi(a|s;\theta) \delta_t   
  \label{eq:nac_actor_update_2}
\end{equation}

\subsection{Python Code for implementing Actor-Critic Algorithm}
The python code for creating actor and critic classes are provided in the code listing
\ref{lst:nac_actor} and \ref{lst:nac_critic} respectively. This code can be used to solve problems
with discrete action spaces only. The actor and critic classes use a sequential deep network to
estimate policy $\pi(s)$ and value function $V(s)$ respectively. The actor network returns the
action probabilities for a given input state $s$.  The actor class also provides a method to
computer actor loss from the TD error and action log probabilities as explained in Section
\ref{sec:nac_adv}. The negative sign in the loss function indicates that the actor is optimized by
maximizing this loss function.  The code for the actor-critic agent is provided in the code listing
\ref{lst:nac_agent}. It provides the \texttt{train()} method to update both policy and critic
parameters after each step of episode. 

\begin{listing}
	\begin{pygments}[frame=single, indent=L]{python}
class Actor():
	def __init__(self, obs_shape, action_size, lr=1e-4, model=None):
		self.obs_shape = obs_shape
		self.action_size = action_size
		self.lr = lr
		if model is None:
		self.model = self._build_model()
		else:
		self.model = model 
		self.optimizer = tf.keras.optimizers.Adam(
               learning_rate=self.lr)


	def _build_model(self): # outputs action probabilities
		sinput = tf.keras.layers.Input(shape=self.obs_shape)
		x = tf.keras.layers.Dense(512, activation='relu',
		   kernel_initializer=tf.keras.initializers.HeUniform())(sinput)
		x = tf.keras.layers.Dense(512, activation='relu',
		   kernel_initializer=tf.keras.initializers.HeUniform())(x)
		aout = tf.keras.layers.Dense(self.action_size, 
         activation='softmax',
		   kernel_initializer=tf.keras.initializers.HeUniform())(x)
		model = tf.keras.models.Model(sinput, aout, name='actor')
		model.summary()
		return model

	def __call__(self, states):
		# returns action probabilities for each state
		pi = tf.squeeze(self.model(states))
		return pi

	def compute_actor_loss(self, states, actions, td_error):
		pi = self.model(states)
		action_dist = tfp.distributions.Categorical(
            probs=pi, dtype=tf.float32)
		log_prob = action_dist.log_prob(actions)
		actor_loss = -log_prob * td_error
		return tf.math.reduce_mean(actor_loss)				
	  \end{pygments}
	\caption{Python code for creating Actor Class}
	\label{lst:nac_actor}
\end{listing}

\begin{listing}
	\begin{pygments}[frame=single, indent=L]{python}
class Critic():
    def __init__(self, obs_shape,
                lr = 1e-4, gamma=0.99, model=None):
        self.obs_shape = obs_shape
        self.gamma = gamma
        self.lr = lr
        if model is None:
            self.model = self._build_model()
        else:
            self.model = model
        self.optimizer = tf.keras.optimizers.Adam(learning_rate=self.lr)


    def _build_model(self): # returns V(s)
        sinput = tf.keras.layers.Input(shape=self.obs_shape)
        x = tf.keras.layers.Dense(128, activation='relu')(sinput)
        x = tf.keras.layers.Dense(256, activation='relu')(x)
        x = tf.keras.layers.Dense(256, activation='relu')(x)
        vout = tf.keras.layers.Dense(1, activation='relu')(x) 
        model = tf.keras.models.Model(inputs= sinput, outputs=vout, 
                                      name='critic')
        model.summary()
        return model

    def __call__(self, states):
        # returns V(s) for each state
        value = tf.squeeze(self.model(states))
        return value
	\end{pygments}
	\caption{Python code for creating Critic Class}
	\label{lst:nac_critic}
\end{listing}

\begin{listing}
  \begin{pygments}[frame=single, indent=L]{python}
class ACAgent():
    def __init__(self, obs_shape, action_size, 
                lr_a=1e-4, lr_c=1e-4, gamma=0.99,
                a_model=None, c_model=None):
        self.obs_shape = obs_shape
        self.action_size = action_size
        self.gamma = gamma
        self.lr_a = lr_a
        self.lr_c = lr_c
        self.name = 'actor-critic'

        # actor model
        self.actor = Actor(self.obs_shape, self.action_size,
                          lr=self.lr_a, model=a_model)
        # critic model
        self.critic = Critic(self.obs_shape, lr=self.lr_c, 
                             gamma=self.gamma, model=c_model)

    def policy(self, state):
        state = tf.expand_dims(
            tf.convert_to_tensor(state, dtype=tf.float32), axis=0)
        pi = self.actor(state) # action probabilities
        action_dist = tfp.distributions.Categorical(
                              probs=pi, dtype=tf.float32)
        action = action_dist.sample()
        return int(action.numpy())


    def train(self, state, action, reward, next_state, done):
        state = tf.expand_dims(
               tf.convert_to_tensor(state, dtype=tf.float32), axis=0)
        action= tf.convert_to_tensor(action, dtype=tf.float32)
        reward = tf.convert_to_tensor(reward, dtype=tf.float32)
        next_state = tf.expand_dims(
               tf.convert_to_tensor(next_state, dtype=tf.float32), 
               axis=0)
        done = tf.convert_to_tensor(done, dtype=tf.float32)
        with tf.GradientTape() as tape1, tf.GradientTape() as tape2:
            value = self.critic(state)
            next_value = self.critic(next_state)
            td_target = reward + self.gamma * next_value * (1 - done)
            td_error = td_target - value
            a_loss = self.actor.compute_actor_loss(state, action, td_error)
            c_loss = tf.math.reduce_mean(tf.square(td_target - value))

        actor_grads = tape1.gradient(a_loss, 
                     self.actor.model.trainable_variables)
        critic_grads = tape2.gradient(c_loss, 
                     self.critic.model.trainable_variables)
        self.actor.optimizer.apply_gradients(zip(actor_grads, 
                              self.actor.model.trainable_variables))
        self.critic.optimizer.apply_gradients(zip(critic_grads, 
                           self.critic.model.trainable_variables))
        return a_loss, c_loss
  \end{pygments}
  \caption{Python Code for Actor-Critic Agent Class}
    \label{lst:nac_agent}
\end{listing}

\subsubsection{Solving CartPole problem using Naive Actor-Critic Algorithm}
The Python function for training a Gym environment problem using a naive actor-critic agent is
provided in code Listing \ref{lst:nac_train}. The main function that uses this train function to
solve \texttt{CartPole-v1} environment is provided in code Listing \ref{lst:nac_cpv1} along with the
console output. The corresponding training performance plot is shown in Figure \ref{fig:nac_cpv1}.
It can be seen that the episodic reward is steadily increasing with training episodes. This problem
is considered solved when the episodic reward crosses 499. So, the problem gets solved in about 400
episodes with best score graph crossing this mark.

\begin{listing}
  \begin{pygments}[frame=single, indent=L]{python}
def train(env, agent, max_episodes=10000, log_freq=50, 
            min_score=-200, max_score=200,
             stop_score=200):

    print('Environment name: ', env.spec.id)
    print('RL Agent name:', agent.name)
    
    assert isinstance(env.action_space, gym.spaces.Discrete),\
                "AC Agent only for discrete action spaces"

    ep_scores = []
    best_score = -np.inf
    for e in range(max_episodes):
        done = False
        state = env.reset()[0]
        ep_score = 0
        a_losses, c_losses = [], []
        while not done:
            action = agent.policy(state)
            next_state, reward, done, _, _ = env.step(action)
            a_loss, c_loss = agent.train(state, action, 
                                 reward, next_state, done)
            a_losses.append(a_loss)
            c_losses.append(c_loss)
            state = next_state
            ep_score += reward
        # while loop ends here
        ep_scores.append(ep_score)

        if e % log_freq == 0:
            print(f'e:{e}, ep_score:{ep_score:.2f}, 
                     avg_ep_score:{np.mean(ep_scores):.2f},\
                        avg100score:{np.mean(ep_scores[-100:]):.2f}, \
                           best_score:{best_score:.2f}')

        if ep_score > best_score:
            best_score = ep_score
            agent.save_weights()
            print(f'Best Score: {ep_score}, episode: {e}. Model saved.')

        if np.mean(ep_scores[-100:]) > stop_score:
            print('The problem is solved in {} episodes'.format(e))
            break
    # for loop ends here
  \end{pygments}
  \caption{Function for training an actor-critic agent for a given Gym environment}
  \label{lst:nac_train}
\end{listing}


\begin{listing}
  \begin{pygments}[frame=single, indent=L]{python}
import gymnasium as gym
from actor_critic import ACAgent 
if __name__ == '__main__':
    env = gym.make('CartPole-v1')
    obs_shape = env.observation_space.shape
    action_size = env.action_space.n

    print("Observation shape: ", obs_shape)
    print("Action Size: ", action_size)
    print("Max Episode steps: ", env.spec.max_episode_steps)

    # create an RL agent
    agent = ACAgent(obs_shape, action_size)

    # train the RL agent on
    train(env, agent, max_episodes=1500, log_freq=100, 
             stop_score=499)
  \end{pygments}
  \begin{framed}
    \begin{scriptsize}
    \begin{verbatim}
Output:
Best score:67.0, episode:0. Model Saved!
e:0, ep_score:67.00, avg_ep_score:67.00, avg100score:67.00, best_score:67.00
Best score:96.0, episode:29. Model Saved!
e:100, ep_score:38.00, avg_ep_score:34.75, avg100score:34.43, best_score:96.00
Best score:190.0, episode:129. Model Saved!
e:200, ep_score:46.00, avg_ep_score:46.21, avg100score:57.78, best_score:190.00
Best score:199.0, episode:216. Model Saved!
Best score:243.0, episode:234. Model Saved!
Best score:347.0, episode:263. Model Saved!
Best score:387.0, episode:274. Model Saved!
e:300, ep_score:124.00, avg_ep_score:76.65, avg100score:137.85, best_score:387.00
Best score:863.0, episode:316. Model Saved!
e:400, ep_score:134.00, avg_ep_score:102.75, avg100score:181.29, best_score:863.00
e:500, ep_score:201.00, avg_ep_score:125.70, avg100score:217.75, best_score:863.00
e:600, ep_score:130.00, avg_ep_score:141.93, avg100score:223.20, best_score:863.00
e:700, ep_score:200.00, avg_ep_score:159.75, avg100score:266.90, best_score:863.00
    \end{verbatim}
  \end{scriptsize}
  \end{framed}
  \caption{Python code for training an actor-critic agent to solve the \texttt{CartPole-v1}
  problem.}
    \label{lst:nac_cpv1}
\end{listing}

\begin{figure}[htbp]
  \centering
  \includegraphics[scale=0.1]{./figures/chap07/cartpole_v1_nac.png}
  \caption{Performance of Naive Actor Critic Algorithm in solving \texttt{CartPole-v1} problem.}
  \label{fig:nac_cpv1}
\end{figure}

\subsubsection{Solving Lunar Lander Problem using Naive Actor-Critic Algorithm}

\begin{listing}
  \begin{pygments}[frame=single, indent=L]{python}
import gymnasium as gym
from actor_critic import ACAgent 
if __name__ == '__main__':
    env = gym.make('LunarLander-v3')
    obs_shape = env.observation_space.shape
    action_size = env.action_space.n

    print("Observation shape: ", obs_shape)
    print("Action Size: ", action_size)
    print("Max Episode steps: ", env.spec.max_episode_steps)

    # create an RL agent
    agent = ACAgent(obs_shape, action_size)

    # train the RL agent on
    train(env, agent, max_episodes=1500, min_score=-500, log_freq=100, 
             stop_score=200)
  \end{pygments}
  \begin{framed}
    \begin{verbatim}
    Output:
    provide console output
    
    \end{verbatim}
  \end{framed}
  \caption{Python code for training a actor-critic agent to solve the \texttt{LunarLander-v3}
  problem.}
    \label{lst:nac_ll}
\end{listing}


\section{Advantage Actor-Critic (A2C) Algorithm} \label{sec:a2c}
\index{Actor-Critic!Advantage Actor-Critic (A2C)}
In Advantage actor-critic algorithm, Advantage function $A(s,a)$ is used directly to update the
actor parameters instead of using raw TD error values which is more noisy. If TD error is used as an estimate
of the advantage function as done in the previous section, there is no particular difference between
A2C and naive actor-critic algorithm discussed above. To encourage exploration and prevent premature
convergence to suboptimal policies, an entropy term is often added to the actor loss given by:

\begin{equation}
  H(\pi_\theta) = - \sum_a \pi_\theta(a|s_t)\log \pi_\theta(a|s_t)
  \label{eq:entropy}
\end{equation}

Therefore the combined actor loss becomes:
\begin{equation}
  L_{actor}(\theta) = -\log\pi_\theta(a_t|s_t).A(s_t, a_t) - \beta H(\pi_\theta)
  \label{eq:a2c_actor_loss}
\end{equation}

Rather than updating the actor and critic parameters after each episode, the training updates can be
carried out in a batch mode by collecting trajectories for a fixed number of time steps. The
pseudocode of A2C algorithm is provided in Algorithm \ref{alg:a2c}.

\begin{algorithm}[htbp]
  \caption{Advantage Actor-Critic (A2C) Algorithm}
	\label{alg:a2c}
	\begin{algorithmic}[1]
     \scriptsize
		\State input: Policy parameters $\theta$, value function parameters $w$. Learning rates
      $\alpha_\theta$, $\alpha_w$, discount factor $\gamma$ and entropy coefficient $\beta$.
      \For {each iterative step}
         \State Collect trajectories with transitions ${(s_t, a_t, r_t, s_{t+1}, d)}$ for
         $t=0,1,\ldots, T$ with $a_t \sim \pi(s_t;\theta)$. 
      \For {each trajectory (or episode)}
      \State Compute Advantage: 
         \begin{equation}
           A(s_t,a_t) = r_{t+1} + \gamma V(s_{t+1};w) * (1-d)-V(s_t;w)
           \label{eq:adv_tde}
         \end{equation}
         \Comment{$A(s_t,a_t) \approx \delta_t$, TD error }
         \State Update Critic: 
         \begin{equation}
           w \leftarrow w + \alpha_w \nabla_w V(s_t;w) A(s,a)
           \label{eq:a2c_c_update}
         \end{equation}

         \Comment{Critic minimises TD error}
		   \State Update Actor: 
         \begin{equation}
           \theta \leftarrow \theta + \alpha_\theta \nabla_\theta \log
            \pi(a|s;\theta) A(s,a) + \beta H(\pi_\theta(s_t)) 
           \label{eq:a2c_a_update}
         \end{equation}
         \Comment{maximises actor loss function \eqref{eq:a2c_actor_loss}}
      \EndFor
		\EndFor
	\end{algorithmic}
\end{algorithm}



\subsection{Python Implementation of A2C Algorithm}
The \texttt{A2CAgent} class implements the algorithm shown in the Algorithm Table \ref{alg:a2c}. The
actor and critic classes remain same as with the previous naive actor-critic implementation.  The
method \texttt{compute\_advantages()} uses TD error to estimate the advantage function from the
state transition information $\{s_t, a_t, r_t, s_{t+1},d\}$ collected during exploration as shown in
equation \eqref{eq:adv_tde}. The method \texttt{compute\_actor\_loss()} computes the combined loss
function given by equation \eqref{eq:a2c_actor_loss}. This function uses
\texttt{tensorflow\_probability} package to compute log probability and entropy values.
The \texttt{train()} method computes advantages, actor and critic loss functions for each episode
and updates the actor and critic parameters using built-in optimizer. The negative sign with the
actor loss function indicates that the actor loss is maximized contrast to the critic loss which is
minimized.  The complete python code for implementing \texttt{A2CAgent} class is provided in the
listing \ref{lst:a2c_agent}. 

It is observed that computing TD error as the difference between the discounted return and the current
value estimate provides better performance compared to the standard method that requires storing
both $s_t$ and $s_{t+1}$. The method to compute discounted rewards and actor loss function is
provided in the code listing \ref{lst:tde_dr}.  

The method \texttt{a2c\_train()} provided in code listing \ref{lst:a2c_train} shows how one can
train an A2C agent on a given gym environment. As we can see in this code, the agent is trained
after each episode indicating a batch mode of training unlike the implementation in the previous
section that updated the models after each step. 

\begin{listing}
  \begin{pygments}[frame=single, indent=L]{python}
import tensorflow_probability as tfp
class A2CAgent:
    def __init__(self, obs_shape, action_size, 
                 lr_a=1e-4, lr_c=1e-3, gamma=0.99,
                 entropy_beta=0.01, grad_clip_norm=5.0,
                 a_model=None, c_model=None):
        self.lr_a = lr_a
        self.lr_c = lr_c
        self.entropy_beta = entropy_beta
        self.grad_clip_norm = grad_clip_norm
        
        self.gamma = gamma
        self.action_size = action_size
        self.obs_shape = obs_shape
        self.name = 'A2C_v2'

        # create actor and critic networks
        self.actor = Actor(obs_shape, action_size, lr=self.lr_a,
                           model=a_model)
        self.critic = Critic(obs_shape, lr=self.lr_c,
                             gamma=self.gamma, model=c_model)

    def policy(self, state):
        state = tf.expand_dims(
            tf.convert_to_tensor(state, dtype=tf.float32), 
               axis=0)
        pi = self.actor(state)
        pi_np = pi.numpy()
        action_probs = tfp.distributions.Categorical(probs=pi)
        action = action_probs.sample()
        return action.numpy()
    
    def compute_advantages(self, rewards, values, next_values, dones):
        advantages = rewards + \
            self.gamma * next_values * (1.0 - dones) - values
        # normalize advantages
        advantages = (advantages - np.mean(advantages)) 
                     / (np.std(advantages) + 1e-8)
        return advantages

    def compute_actor_loss(self, states, actions, advantages):
        probs = self.actor(states)
        action_dist = tfp.distributions.Categorical(
                           probs=probs, dtype=tf.float32)
        log_probs = action_dist.log_prob(actions)
        actor_loss = -tf.reduce_mean(log_probs * advantages)

        # entropy regularization
        entropy = action_dist.entropy() 
        actor_loss -= self.entropy_beta * tf.reduce_mean(entropy)
        return actor_loss

        
    def train(self, states, actions, rewards, next_states, dones):
        states = np.array(states, dtype=np.float32)
        actions = np.array(actions, dtype=np.int32)
        rewards = np.array(rewards, dtype=np.float32)
        next_states = np.array(next_states, dtype=np.float32)
        dones = np.array(dones, dtype=np.float32)

        with tf.GradientTape() as tape1, tf.GradientTape() as tape2:
            values = self.critic(states)
            next_values = self.critic(next_states)
            advantages = self.compute_advantages(
                        rewards, values, next_values, dones)
            actor_loss = self.compute_actor_loss(states, 
                                             actions,advantages)
            critic_loss = tf.reduce_mean(tf.square(advantages))

        # compute gradients 
        actor_grads = tape1.gradient(actor_loss, 
                     self.actor.model.trainable_variables)
        critic_grads = tape2.gradient(critic_loss, 
                     self.critic.model.trainable_variables)

        # apply gradients to the models
        self.actor.optimizer.apply_gradients(
            zip(actor_grads, self.actor.model.trainable_variables))
        self.critic.optimizer.apply_gradients(
            zip(critic_grads, self.critic.model.trainable_variables))
        return actor_loss.numpy(), critic_loss.numpy()
  \end{pygments}
  \caption{Python Code for implementing A2C agent class}
  \label{lst:a2c_agent}
\end{listing}


\begin{listing}
  \begin{pygments}[frame=single, indent=L]{python}
class A2CAgent():
    def init(self):
        pass

    def policy(self, state):
        pass

    def compute_discounted_rewards(self, states, actions, rewards):
        discounted_rewards = []
        sum_rewards = 0
        for r in reversed(rewards):
            sum_rewards = r + self.gamma * sum_rewards
            discounted_rewards.insert(0, sum_rewards)
        discounted_rewards = np.array(discounted_rewards)
        states = np.array(states, dtype=np.float32)
        actions = np.array(actions, dtype=np.int32)
        return states, actions, discounted_rewards


    def compute_actor_loss(self, states, actions, td_error):
        probs = self.actor(states)
        action_dist = tfp.distributions.Categorical(
               probs=probs, dtype=tf.float32)
        log_probs = action_dist.log_prob(actions)
        actor_loss = -tf.reduce_mean(log_probs * td_error)
        # entropy regularization
        entropy = action_dist.entropy() 
        actor_loss -= self.entropy_beta * tf.reduce_mean(entropy)
        return actor_loss

    def train(self, states, actions, rewards):
        states, actions, discnt_rewards = \
              self.compute_discounted_rewards(states, actions, rewards)
        discnt_rewards = tf.convert_to_tensor(
                           discnt_rewards, dtype=tf.float32)

        with tf.GradientTape() as tape1, tf.GradientTape() as tape2:
            values = self.critic(states)
            td_error = tf.math.subtract(discnt_rewards, values)
            actor_loss = self.actor.compute_actor_loss(states, 
                                                actions, td_error)
            critic_loss = tf.reduce_mean(tf.square(td_error))

        actor_grads = tape1.gradient(actor_loss, 
                     self.actor.model.trainable_variables)
        critic_grads = tape2.gradient(critic_loss, 
                     self.critic.model.trainable_variables)

        self.actor.optimizer.apply_gradients(
                  zip(actor_grads, self.actor.model.trainable_variables))
        self.critic.optimizer.apply_gradients(
                  zip(critic_grads, self.critic.model.trainable_variables))
        return actor_loss.numpy(), critic_loss.numpy()

  \end{pygments}
  \caption{Code for computing TD error from discounted returns}
  \label{lst:tde_dr}
\end{listing}

\begin{listing}
  \begin{pygments}[frame=single, indent=L]{python}
def a2c_train(env, agent, max_episodes=10000, log_freq=50, 
               max_score=None, min_score=None, 
               stop_score=500):
    print('Environment name: ', env.spec.id)
    print('RL Agent name:', agent.name)
    
    assert isinstance(env.action_space, gym.spaces.Discrete),\
                "A2C Agent only for discrete action spaces"

    ep_scores = []
    best_score = -np.inf
    for e in range(max_episodes):
        states, actions, rewards = [], [], []
        done = False
        state = env.reset()[0]
        ep_score = 0
        while not done:
            action = agent.policy(state)
            next_state, reward, done, _, _ = env.step(action)
            rewards.append(reward)
            states.append(state)
            actions.append(action)
            state = next_state
            ep_score += reward

            if max_score is not None and 
                     ep_score >= max_score:
                done = True
            if min_score is not None and 
                     ep_score <= min_score:
                done = True

            if done: 
                ep_scores.append(ep_score)
                # train the agent
                a_loss, c_loss = agent.train(states, actions, rewards)

        # while loop ends here

        if e % log_freq == 0:
            print(f'e:{e}, ep_score:{ep_score:.2f},\ 
              avg_ep_score:{np.mean(ep_scores):.2f},\
               avg100score:{np.mean(ep_scores[-100:]):.2f}, \
                best_score:{best_score:.2f}')
        
        if ep_score > best_score:
            best_score = ep_score
            agent.save_weights()
            print(f'Best Score: {ep_score}, episode: {e}. Model saved.')

    # for loop ends here
  \end{pygments}
  \caption{Training function for an A2C agent}
  \label{lst:a2c_train}
\end{listing}
\subsection{Solving \texttt{LunarLander-v3} with A2C algorithm}

The main code for training an A2C agent to solve Gymnasium's \texttt{LunarLander-v3} problem is
provided in the listing \ref{lst:ll_a2c}. The actor and critic networks are created externally and
passed to the \texttt{A2CAgent} object. To speed up training, the episodes are truncated when the
total episodic score goes beyond the range $[-300, 500]$. The resulting training performance plot is
shown in Figure \ref{fig:a2c_llv3}. It can be seen that the best score exceeds 200 which is required
for solving the problem successfully. However, the running average of last 100 episodes reaches
around 0 after about 1500 episode. 

\begin{listing}
  \begin{pygments}[frame=single, indent=L]{python}
import gymnasium as gym 
# create actor & Critic models
def create_actor_model(obs_shape, n_actions):
    s_input = tf.keras.layers.Input(shape=obs_shape)
    x = tf.keras.layers.Dense(128, activation='relu')(s_input)
    x = tf.keras.layers.Dense(128, activation='relu')(x)
    a = tf.keras.layers.Dense(n_actions, activation='softmax')(x)
    model = tf.keras.models.Model(s_input, a, name='actor_network')
    model.summary()
    return model

def create_critic_model(obs_shape):
    s_input = tf.keras.layers.Input(shape=obs_shape)
    x = tf.keras.layers.Dense(128, activation='relu')(s_input)
    x = tf.keras.layers.Dense(128, activation='relu')(x)
    v = tf.keras.layers.Dense(1, activation=None)(x)
    model = tf.keras.models.Model(s_input, v, name='critic_network')
    model.summary()
    return model

if __name__ == '__main__':
    # Create Gym environment 
    env = gym.make('LunarLander-v3', render_mode='rgb_array')
    obs_shape = env.observation_space.shape
    action_size = env.action_space.n

    print("Observation shape: ", obs_shape)
    print("Action Size: ", action_size)
    print("Max Episode steps: ", env.spec.max_episode_steps)

    actor_net = create_actor_model(obs_shape, action_size)
    critic_net = create_critic_model(obs_shape)

    # create an RL agent
    agent = A2CAgent(obs_shape, action_size, 
                   a_model=actor_net,  c_model=critic_net)

    # train the RL agent on
    a2c_train(env, agent, max_episodes=1500, log_freq=100, 
         stop_score=200, max_score=500, min_score=-300)
  \end{pygments}
  \begin{framed}
    \begin{scriptsize}
    \begin{verbatim}
Best score:-286.7096895045363, episode:0. Model Saved!
e:0, ep_score:-286.71, avg_ep_score:-286.71, avg100score:-286.71, best_score:-286.71
Best score:-278.85662087282384, episode:2. Model Saved!
Best score:7.116470473337458, episode:3. Model Saved!
Best score:23.33927428594974, episode:10. Model Saved!
Best score:100.85838120915994, episode:20. Model Saved!
Best score:166.68768124754476, episode:21. Model Saved!
Best score:167.36705054863853, episode:31. Model Saved!
Best score:211.82154501357178, episode:41. Model Saved!
Best score:221.82705954781403, episode:49. Model Saved!
Best score:236.63119714607453, episode:57. Model Saved!
Best score:249.53318413060276, episode:71. Model Saved!
Best score:255.81425819686874, episode:72. Model Saved!
e:100, ep_score:-133.23, avg_ep_score:-83.00,  avg100score:-80.96, best_score:255.81
e:200, ep_score:-121.24, avg_ep_score:-109.77, avg100score:-136.81, best_score:255.81
e:300, ep_score:-144.85, avg_ep_score:-116.23, avg100score:-129.22, best_score:255.81
e:400, ep_score:-121.21, avg_ep_score:-120.69, avg100score:-134.12, best_score:255.81
e:500, ep_score:-124.12, avg_ep_score:-122.79, avg100score:-131.17, best_score:255.81
e:600, ep_score:-113.23, avg_ep_score:-124.91, avg100score:-135.57, best_score:255.81
e:700, ep_score:-180.10, avg_ep_score:-124.89, avg100score:-124.78, best_score:255.81
e:800, ep_score:-157.82, avg_ep_score:-125.42, avg100score:-129.12, best_score:255.81
e:900, ep_score:-108.00, avg_ep_score:-126.25, avg100score:-132.86, best_score:255.81    
    \end{verbatim}
  \end{scriptsize}
  \end{framed}
  \caption{Main function for applying A2C agent to solve \texttt{LunarLander-v3} problem}
  \label{lst:ll_a2c}
\end{listing}

\begin{figure}[htbp]
  \centering
  \includegraphics[scale=0.1]{./figures/chap07/lunarlander_v3_a2c.png}
  \caption{Training performance of A2C algorithm for \texttt{Lunarlander-v3} environment}
  \label{fig:a2c_llv3}
\end{figure}

\section{Asynchronous Advantage Actor-Critic (A3C) Model} \label{sec:a3c}
\index{Actor-Critic!Asynchronous Advantage Actor-Critic (A3C)}
In A2C, a single agent interacts with the environment, collects a batch of experience (e.g., a few
steps or a full episode), computes the losses for both actor and critic based on this batch, and
then updates the agent's network parameters. It is also possible to have
multiple workers to collect experiences from different parallel environments and
performs a single coordinate update to the shared network parameters of the agent.

\gls{a3c} \cite{sewak2019actor} was a breakthrough in deep reinforcement learning, primarily because
it enabled highly efficient training on multi-core CPUs without requiring a replay buffer (like in
DQN). The "Asynchronous" aspect is key. A3C maintains a global network (actor and critic) and
multiple "worker" agents. Each worker agent has its own copy of the network parameters and interacts
with its own independent copy of the environment. Each of the multiple parallel worker agents
running in separate threads collects its own experience tuples $(s,a,r,s')$  independently.
Crucially, each worker computes gradients for its local network parameters based on its collected
experience. Instead of waiting for other workers, it asynchronously sends these gradients to the
global network to update the global parameters. Periodically, or after a certain number of steps,
each worker agent synchronizes its local network parameters with the updated global network
parameters. This ensures that all workers are learning from the collective experience without being
perfectly in sync. The schematic block diagram of A3C architecture is shown in Figure
\ref{fig:a3c_bd}.  

The mathematical equations for the actor and critic losses are fundamentally the same as in A2C, as
they both use the advantage function. The difference lies in how the updates are performed.

\begin{figure}[htbp]
  \centering
  \includegraphics[scale=0.3]{./figures/chap07/a3c_bd.png}
  \caption{Block Diagram to understand A3C architecture}
  \label{fig:a3c_bd}
\end{figure}

\subsection{Python Implementation of A3C Model}

\subsubsection{The \texttt{worker} function}

The Python code for implementing a worker is provided in the listing \ref{lst:a3c_worker}. The
\texttt{worker} function creates a local \texttt{A2CAgent} referred to as \texttt{local\_network}
using the class definition provided in the previous section. It also creates its own gym environment
for interaction. During each episode, the \texttt{local\_network} starts by copying the weights from
the \texttt{global\_network} and uses these parameters to collects experience tuples $(s, a, r,
s')$. It then computes gradients at the end of the each episode by using this batch of experiences.
The computed gradients are pushed to the \texttt{gradients\_queue} which can then be used for
updating parameters of the \texttt{global\_network}. The parameters between multiple parallel
processes are shared by using queues. Several steps are taken to ensure that the queues are not
empty leading to deadlock situations. This includes pushing more data into the queue and taking
appropriate steps if the queue is empty or full etc. It is also necessary to clear the memory after
each thread to prevent running out of memory on the computer. This code uses Python's
multiprocessing library to run the code on a multi-core CPU and does not require GPUs for execution.


\begin{listing}
  \begin{pygments}[frame=single, indent=L]{python}
import gymnasium as gym
import numpy as np
import multiprocessing
import time
import gc
import queue
from a2c import A2CAgent
from collections import deque

# Disable GPU for this script
os.environ['CUDA_VISIBLE_DEVICES'] = '-1'  

import tensorflow as tf
import tensorflow_probability as tfp

# Worker function for each process
def worker(worker_id, global_weights_queue, 
           gradients_queue, save_request_queue, env_id, 
           create_actor_func=None,
           create_critic_func=None,
           max_episodes=1500, max_score = 500, 
           min_score = -500, max_steps = None):

    # Set random seed for reproducibility in each process
    tf.random.set_seed(worker_id + 1)
    np.random.seed(worker_id + 1)

    # create environment
    env = gym.make(env_id)
    obs_shape = env.observation_space.shape
    action_size = env.action_space.n

    # Create local network and environment
    if create_actor_func is not None:
        actor = create_actor_func(obs_shape, action_size)
    else:
        actor = None

    if create_critic_func is not None:
        critic = create_critic_func(obs_shape)
    else:
        critic = None
    local_network = A2CAgent(obs_shape, action_size,
                             a_model=actor, c_model=critic)

    # collect experience tuple and compute gradients
    # for the local network
    episode = 0
    ep_scores = [] 
    best_score = -np.inf
    for episode in range(max_episodes):
        weights_updated = False
        for _ in range(3):  # Retry up to 3 times
            try:
                global_weights = global_weights_queue.get_nowait()
                local_network.set_weights(*global_weights)
                weights_updated = True
                break # retrieval success
            except queue.Empty:
                time.sleep(0.1)  # Brief pause before retry
        if not weights_updated: 
            continue # Continue with current weights if queue is empty

        state = env.reset()[0]
        episode_reward = 0
        done = False
        step = 0
        states = deque(maxlen=max_steps \
               if max_steps is not None else 1000) 
        actions = deque(maxlen=max_steps \
               if max_steps is not None else 1000) 
        rewards = deque(maxlen=max_steps \
               if max_steps is not None else 1000)

        # Collect trajectory
        while not done: 
            action = int(local_network.policy(state))
            next_state, reward, done, truncated, _ = env.step(action)
            done = done or truncated

            states.append(state) # type: ignore
            actions.append(action)
            rewards.append(reward)

            state = next_state
            episode_reward += reward

            step += 1

            if max_score is not None and episode_reward >= max_score:
                done = True
            if min_score is not None and episode_reward <= min_score:
                done = True
            if max_steps is not None and step >= max_steps:
                done = True

        # end of episode
        ep_scores.append(episode_reward)

        # update the local network 
        a_loss, c_loss, actor_grads, critic_grads = \
               local_network.compute_gradients(
            states, actions, rewards 
        )

        # Update global network
        try:
            gradients_queue.put_nowait((actor_grads, critic_grads))
        except queue.Full: 
            continue # queue full, skip updated


        if episode_reward > best_score:
            best_score = episode_reward

        try:
            save_request_queue.put_nowait(
                  (worker_id, episode, episode_reward))
        except queue.Full:  # skip save request
            continue

        # free memory  
        tf.keras.backend.clear_session()  # Clear TensorFlow session
        states.clear()
        actions.clear()
        rewards.clear()
        gc.collect()  # Collect garbage to free memory
    # end of for-loop
    env.close()
  \end{pygments}
  \caption{Worker function that creates an agent to collect experiences from its own individual environment and
    computes gradient. These gradients are then sent to the global network asynchronously for update}
  \label{lst:a3c_worker}
\end{listing}

\subsubsection{Running multiple workers}

The code for running multiple workers on parallel threads is provided in the listing
\ref{lst:a3c_runw}. The function \texttt{run\_workers()} uses Python's multiprocessing
module to achieve parallel execution of worker function. It creates a global \texttt{A2CAgent}
referred to as \texttt{global\_networks} whose parameters are shared with the local network of each
worker through the multiprocessing queue called \texttt{global\_weights\_queue}. The gradients
computed by the local networks are received by the global network by using another multiprocessing
queue called \texttt{gradients\_queue}. The gradients received are averaged and then applied to
update the parameters of the \texttt{global\_network}. Adequate steps are taken to deal with
situations when multi-processing queues are not empty or full. This prevents deadlock situation
where the processes wait indefinitely to receive data from empty queues. 


\begin{listing}
  \begin{pygments}[frame=single, indent=L]{python}
def run_workers(env_name, max_num_workers=5, max_episodes=1500, 
         wandb_log=True, max_score=500, min_score=-200,
         max_steps=1000,
         create_actor_func=None,
         create_critic_func=None):
    # Set random seed for reproducibility
    tf.random.set_seed(42)
    np.random.seed(42)

    N_WORKERS = min(multiprocessing.cpu_count(), max_num_workers)
    print('N_WORKERS:', N_WORKERS)


    # Initialize environment to get state and action sizes
    env = gym.make(env_name)

    obs_shape = env.observation_space.shape
    action_size = env.action_space.n
    env_id = env.spec.id
    env.close()

    if create_actor_func is not None: 
        a_model = create_actor_func(obs_shape, action_size)
    else:
        a_model = None
    if create_critic_func is not None:
        c_model = create_critic_func(obs_shape)
    else:
        c_model = None

    # Initialize global network
    global_network = A2CAgent(obs_shape, action_size,
                              a_model=a_model, c_model=c_model)

    # Create a queue to share weights between processes
    manager = multiprocessing.Manager()
    global_weights_queue = manager.Queue(maxsize=2*N_WORKERS)
    gradients_queue = manager.Queue(maxsize=2*N_WORKERS)
    save_request_queue = manager.Queue(maxsize=N_WORKERS)
    save_lock = manager.Lock()

    # initial weights for the global network
    for _ in range(2*N_WORKERS):
        # Initialize queue with the global network's weights
        global_weights_queue.put(global_network.get_weights())


    # Create and start worker processes
    processes = []
    for i in range(N_WORKERS):
        p = multiprocessing.Process(
            target=worker,
            args=(i, global_weights_queue, gradients_queue, 
                  save_request_queue, env_id, 
                  create_actor_func, create_critic_func,  
                  max_episodes,
                  max_score, min_score, max_steps, wandb_log)
        )
        p.start()
        processes.append(p)
        print(f'Started worker {p.name}')

    try:
        last_refill_time = time.time()
        best_reward = -np.inf
        while(any(p.is_alive() for p in processes)):
            gradients = []
            for _ in range(N_WORKERS):
                if not gradients_queue.empty():
                    gradients.append(gradients_queue.get())

            if gradients: # process gradients if available 
                # Filter out None gradients
                valid_gradients = [
                    (actor_grads, critic_grads)
                    for actor_grads, critic_grads in gradients
                    if actor_grads is not None and critic_grads is not None
                ]
                # Only proceed if there are valid gradients
                if valid_gradients:  
                    actor_grads_avg = []
                    critic_grads_avg = []
                    for actor_grads, critic_grads in valid_gradients:
                        if not actor_grads_avg:
                            actor_grads_avg = [tf.convert_to_tensor(g)\
                                    for g in actor_grads]
                            critic_grads_avg = [tf.convert_to_tensor(g)\
                                    for g in critic_grads]
                        else:
                            for i, g in enumerate(actor_grads):
                                actor_grads_avg[i] = tf.add(
                                    actor_grads_avg[i], tf.convert_to_tensor(g))
                            for i, g in enumerate(critic_grads):
                                critic_grads_avg[i] = tf.add(
                                    critic_grads_avg[i], tf.convert_to_tensor(g))

                    n = len(valid_gradients)
                    actor_grads_avg = [tf.math.truediv(g, n) \
                                       for g in actor_grads_avg]
                    critic_grads_avg = [tf.math.truediv(g, n) \
                                       for g in critic_grads_avg]

                    global_network.apply_gradients(actor_grads_avg, 
                                                critic_grads_avg)
            
                # Synchronize global weights with all workers
                updated_weights = global_network.get_weights()
                for _ in range(N_WORKERS):
                    try:
                        # Synchronize global weights with all workers
                        global_weights_queue.put_nowait(updated_weights)
                    except queue.Full: # skip synchronization
                        while not global_weights_queue.empty():
                            try: 
                                global_weights_queue.get_nowait()
                            except queue.Empty:
                                break
                        global_weights_queue.put_nowait(updated_weights)

            # periodically refill the global weights queue
            if time.time() - last_refill_time > 5:
                try:
                    global_weights_queue.put_nowait(
                              global_network.get_weights())
                    last_refill_time = time.time()
                except queue.Full: # skip refill
                    continue

            # saving weights
            while not save_request_queue.empty():
                try:
                    worker_id, episode, episode_reward = \
                              save_request_queue.get_nowait()
                    print(f"Worker: {worker_id},\ 
                           episode: {episode},  \
                           reward: {episode_reward:.2f}")
                    if episode_reward > best_reward:
                        best_reward = episode_reward
                        with save_lock:
                            try:
                                global_network.save_weights(
                                    actor_wt_file='a3c_actor.weights.h5',
                                    critic_wt_file='a3c_critic.weights.h5',
                                )
                                print(f"Best Score: {best_reward}. \
                                                Saved weights!")
                            except Exception as e:
                                print(f"Error saving weights: {e}")
                except queue.Empty:
                    break
    except Exception as e:
        print(f"An error occurred: {e}")
    finally:
        print("Shutting down workers...")
        #Wait for all processes to complete
        for p in processes:
            p.join()
            print(f'Worker {p.name} has finished.')
  \end{pygments}
  \caption{Python Function to run multiple workers in parallel. The gradients are collected from
  each worker and used for updating the parameters of the global network asynchronously.}
  \label{lst:a3c_runw}
\end{listing}

\subsubsection{Updating \texttt{A2CAgent} to work with multiple workers}

The \texttt{A2CAgent} class declaration provided in Listing \ref{lst:a2c_agent} is updated to
include some additional methods to facilitate working with multiple workers.  This is provided in
Listing \ref{lst:a3c_grad_comp}.  This primarily includes separating the gradient computation from
the parameter update steps. The method \texttt{compute\_gradients()} is used by the
\texttt{local\_network} agent of each worker to compute gradients. The method
\texttt{apply\_gradients()} is used by the \texttt{global\_network} to update its parameters. The
methods \texttt{get\_weights()} and \texttt{set\_weights()} allow copying of parameters between the
global network and the local networks.  Gradient clipping is applied to ensure that the gradients do
not grow out of bounds leading to \texttt{NaN} values.

\begin{listing}
  \begin{pygments}[frame=single, indent=L]{python}
class A2CAgent:
    def compute_gradients(self, states, actions, rewards):
        states, actions, discnt_rewards = \
            self.compute_discounted_rewards(states, actions, rewards)

        with tf.GradientTape() as tape1, tf.GradientTape() as tape2:
            values = self.critic(states)
            td_error = tf.math.subtract(discnt_rewards, values)
            actor_loss = self.compute_actor_loss(states,\ 
                                       actions, td_error)
            critic_loss = tf.reduce_mean(tf.square(td_error))

        # compute gradients 
        actor_grads = tape1.gradient(actor_loss, 
                     self.actor.model.trainable_variables)
        critic_grads = tape2.gradient(critic_loss, 
                     self.critic.model.trainable_variables)

        # clip gradients 
        actor_grads = [tf.clip_by_norm(grad, self.grad_clip_norm) \
                  if grad is not None else None for grad in actor_grads ]
        critic_grads = [tf.clip_by_norm(grad, self.grad_clip_norm) \
                  if grad is not None else None for grad in critic_grads]  

        # Check for NaN gradients
        actor_has_nan = any(tf.reduce_any(tf.math.is_nan(grad)) \
                        for grad in actor_grads if grad is not None)
        critic_has_nan = any(tf.reduce_any(tf.math.is_nan(grad)) \
                        for grad in critic_grads if grad is not None)


        if actor_has_nan or critic_has_nan:
            print(f"NaN gradients detected! "
                  f"Actor NaN: {actor_has_nan}, \
                           Critic NaN: {critic_has_nan}, "
            # Skip sending gradients to avoid corrupting global network
            return None, None, None, None

        return actor_loss.numpy(), critic_loss.numpy(), \
               actor_grads, critic_grads
    
    def apply_gradients(self, actor_grads, critic_grads):
        self.actor.optimizer.apply_gradients(
            zip(actor_grads, self.actor.model.trainable_variables))
        self.critic.optimizer.apply_gradients(
            zip(critic_grads, self.critic.model.trainable_variables))

    def get_weights(self):
        return self.actor.model.get_weights(), \
                     self.critic.model.get_weights()

    def set_weights(self, actor_weights, critic_weights):
        self.actor.model.set_weights(actor_weights)
        self.critic.model.set_weights(critic_weights)
  \end{pygments}
  \caption{Additional methods needed to use \texttt{A2CAgent} with multiple workers. Specifically,
  the gradient computation is separated from the updating parameters stage.}  
    \label{lst:a3c_grad_comp}
\end{listing}


\subsection{Solving \texttt{LunarLander-v3} using A3C model}
The code provided in Listing \ref{lst:a3c_llv3} shows the main code that calls the function
\texttt{run\_workers()} function to train a global \texttt{A2CAgent} by running multiple workers in
parallel. The training performance of the resulting A3C agent is shown in Figure
\ref{fig:a3c_llv3}. It shows how the average of last 100 episodes (\texttt{average100score}) evolves
with increasing training episodes. It can be easily seen that A3C clearly outperforms A2C algorithm
by achieving higher average episodic score over time. This figure also shows two versions A3C
implementation. \texttt{A3C\_v1} uses discounted return to compute TD error while \texttt{A3C\_v2}
uses the standard Bellman equation to compute TD error from the current $s$ and next states $s'$
information. 

\begin{listing}
  \begin{pygments}[frame=single, indent=L]{python}
from a3c import run_workers
import multiprocessing
import tensorflow as tf

# create actor & Critic models
def create_actor_model(obs_shape, n_actions):
    s_input = tf.keras.layers.Input(shape=obs_shape)
    x = tf.keras.layers.Dense(128, activation='relu')(s_input)
    x = tf.keras.layers.Dense(128, activation='relu')(x)
    a = tf.keras.layers.Dense(n_actions, activation='softmax')(x)
    model = tf.keras.models.Model(s_input, a, name='actor_network')
    model.summary()
    return model

def create_critic_model(obs_shape):
    s_input = tf.keras.layers.Input(shape=obs_shape)
    x = tf.keras.layers.Dense(128, activation='relu')(s_input)
    x = tf.keras.layers.Dense(128, activation='relu')(x)
    v = tf.keras.layers.Dense(1, activation=None)(x)
    model = tf.keras.models.Model(s_input, v, name='critic_network')
    model.summary()
    return model

if __name__ == '__main__':
    multiprocessing.set_start_method('spawn')
    run_workers(
        env_name='LunarLander-v3',
        max_num_workers=20,
        max_episodes=1500,
        max_score=500,
        min_score=-200,
        max_steps=1000,
        create_actor_func=create_actor_model,
        create_critic_func=create_critic_model,
    )

  \end{pygments}
  \begin{framed}
    \begin{scriptsize}
      \begin{verbatim}
Worker: 11, episode: 7, reward: -106.48
Worker: 16, episode: 9, reward: -139.72
Worker: 2, episode: 7, reward: -203.49
Worker: 15, episode: 9, reward: -148.49
Worker: 10, episode: 9, reward: 26.72
Best Score: 26.721404883008873. Saved weights!
Worker: 1, episode: 9, reward: -164.57
Worker: 7, episode: 10, reward: -129.38
Worker: 19, episode: 11, reward: -96.75
Worker: 14, episode: 9, reward: -113.13
Worker: 13, episode: 10, reward: -213.46
Worker: 9, episode: 8, reward: -200.68
.. .
.. .
Worker: 15, episode: 439, reward: 132.81
Worker: 5, episode: 436, reward: 165.65
Worker: 16, episode: 443, reward: 87.35
Worker: 4, episode: 435, reward: 133.58
.. .
.. .
orker: 18, episode: 567, reward: 128.30
Worker: 10, episode: 577, reward: 150.79
Worker: 17, episode: 581, reward: 170.52
Worker: 7, episode: 592, reward: 181.43
Worker: 11, episode: 601, reward: 135.22
Worker: 2, episode: 593, reward: 184.58

      \end{verbatim}
    \end{scriptsize}
  \end{framed}
  \caption{Main python code for training A3C agent on \texttt{LunarLander-v3} problem environment}
  \label{lst:a3c_llv3}
\end{listing}

\begin{figure}[htbp]
  \centering
  \begin{tabular}{cc}
  \includegraphics[scale=0.08]{./figures/chap07/lunarlander_v3_a3c.png} & 
  \includegraphics[scale=0.08]{./figures/chap07/lunarlander_v3_a3c_best.png} \\
  {\scriptsize (a) Average of last 100 episodes} & {\scriptsize (b) Best Episodic Score}
\end{tabular}
  \caption{Training performance of A3C algorithm for \texttt{LunarLander-v3} environment}
  \label{fig:a3c_llv3}
\end{figure}

\section{Summary}
This chapter explores actor-critic methods, a cornerstone of reinforcement learning that integrates
policy-based and value-based approaches. It begins with the foundational actor-critic framework,
where the actor learns a policy to select actions and the critic evaluates them by estimating state
values, balancing exploration and exploitation. The naive actor-critic approach is introduced,
highlighting its simplicity but also its limitations, such as high variance in policy updates due to
unscaled rewards. The chapter then delves into advanced variants: A2C (Advantage Actor-Critic) and
A3C (Asynchronous Advantage Actor-Critic). A2C improves stability by using the advantage function to
guide policy updates and synchronizes experience collection across multiple environments for
consistent gradient updates. In contrast, A3C leverages asynchronous parallel workers to update a
shared network, enhancing training speed but introducing potential instability from stale gradients.
Through mathematical formulations and practical comparisons, the chapter illustrates how these
methods address variance and efficiency challenges, making them suitable for complex tasks like game
playing and robotics.
